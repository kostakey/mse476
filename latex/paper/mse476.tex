\documentclass[12pt]{article}
\usepackage[letterpaper, top=1.25in, bottom=1.25in, left=1in, right=1in]{geometry}
\usepackage{fancyhdr} % http://ctan.org/pkg/fancyhdr
\usepackage{hyperref}
\usepackage{graphicx}
\usepackage{indentfirst}
\usepackage[export]{adjustbox}

% % % % % % % % % % % % 
% Citation formatting %
% % % % % % % % % % % % 

\usepackage[style=ieee]{biblatex}
\addbibresource{sources.bib}

% % % % % % % % %
% Hyperlinking  %
% % % % % % % % %

\hypersetup{
    colorlinks=false, % set true if you want colored links
    linktoc=all,     % set to all if you want both sections and subsections linked
    % linkcolor=black,  %c hoose some color if you want links to stand out
    % citecolor=black
    % urlcolor=black
}

% % % % % % % % % % % % % % % %
% Global title, author, date  %
% % % % % % % % % % % % % % % %

\title{Ferrous Metals in Electric Motors}
\author{Kosta Sergakis}
\date{December 1, 2025}
\makeatletter
\let\runauthor\@author
\let\runtitle\@title
\let\rundate\@date
\makeatother

% % % % % % % % % % %
% Header and Footer %
% % % % % % % % % % %

\pagestyle{fancy} % change page style to fancy
\fancyhf{} % clear header/footer
\setlength{\headheight}{15pt}
\fancyhead[L]{Sergakis}
\fancyhead[C]{\runtitle}
\fancyhead[R]{\rundate}
\fancyfoot[C]{\thepage} % \fancyfoot[R]{\thepage}
\renewcommand{\headrulewidth}{0.4pt} % default \headrulewidth is 0.4pt
\renewcommand{\footrulewidth}{0.4pt} % default \footrulewidth is 0pt

% % % % % % % % % %
% Paper Beginning %
% % % % % % % % % %

% \usepackage{natbib}
\begin{document}

    % % % % % % % %
    % Title Page  %
    % % % % % % % %
    
    \begin{titlepage}
        \begin{center}
            \vspace*{1.5cm}
            
            \textbf{Exploring Physical Metallurgy of Ferrous and Aluminum Alloys}
            
            \vspace{1cm}
            
            An exploration of:\\
            \vspace{0.25cm}
            \runtitle
                
            \vspace{1cm}
            
            \textbf{\runauthor}
            
            \vfill  
            
            \vspace{1cm}

            \includegraphics[width=0.3\textwidth]{resources/michigan-state-logo-png-transparent.png}
                
            Chemical Engineering and Materials Science\\
            Michigan State University\\
            East Lansing, Michigan\\
            \rundate
                
        \end{center}
    
    \end{titlepage}


    % % % % % % % % % % %
    % Table of Contents %
    % % % % % % % % % % %

    \setcounter{secnumdepth}{0} % removes section numbers
    % \maketitle
    
    \tableofcontents
    
    \newpage

    \section{Introduction}
        \subsection{Fundamentals of Electric Motors}
            
            \begin{center}
                \vspace{0.5cm}
                \includegraphics[width=0.5\textwidth]{resources/lorentz force.png}
                \\
                \textbf{Figure 1: Lorentz Force} \autocite{Hughes_2011}
                \label{lorentz}
            \end{center}
        
            Electric motors work by converting electrical energy into mechanical motion using the fundamental principles of electromagnetism. In other words, mechanical rotational energy produced through the interaction of magnetic fields and electric currents. At their core, all motors rely on said principle of electromagnetism: when a current-carrying conductor is placed within a magnetic field, it experiences a force (Lorentz force) \hyperref[lorentz]{[Figure 1]}. Electric motors harness this force to generate continuous torque on a rotating element called the rotor, while stationary components such as the stator create structured magnetic fields that drive rotation.
            
            \begin{center}
                \includegraphics[width=0.5\textwidth]{resources/dc commutator.png}
                \\
                \textbf{Figure 2: DC Motor Commutation} \autocite{5562688}
                \label{commutation}
            \end{center}

            \begin{center}
                \includegraphics[width=0.5\textwidth]{resources/ac motor.png}
                \\
                \textbf{Figure 3: AC Motor Rotating Magnetic Field} \autocite{Hughes_2011}
                \label{ac}
            \end{center}

            In a typical motor, the stator generates a magnetic field either through permanent magnets or electromagnetism. The rotor carries windings or magnets that interact with this field. By energizing the stator windings in a timed sequence, the magnetic field rotates, pulling the rotor with it. For DC motors, commutators mechanically switch current direction in the rotor to maintain torque \hyperref[commutation]{[Figure 2]}. In AC motors, the stator's alternating current naturally produces a rotating magnetic field, eliminating mechanical commutation \hyperref[ac]{[Figure 3]}.

        \subsection{Applications Across Different Industries}

            \begin{center}
                \includegraphics[width=0.8\textwidth]{resources/power rating for motor types.png}
                \\
                \textbf{Figure 4: Power Rating for Motor Types} \autocite{Hughes_2011}
                \label{powah}
            \end{center}

            \hyperref[powah]{[Figure 4]}

    \section{Materials Used in Electric Motors}
        \subsection{Material Requirements}
        \subsection{Common Materials}
            \subsubsection{Silicon Steels (Electrical Steels)}
            \subsubsection{Soft Magnetic Composites}
            \subsubsection{Cast Iron and Structural Steels}
    \section{Manufacturing of Electric Motors}
        \subsection{Manufacturing Process}
            \subsubsection{Effects of Temperature and Processing}
            \subsubsection{Stator and Rotor Fabrication}
        % \subsection{Loss Mechanisms in Electric Steels}
    \section{Performance Metrics and Evaluation}
    \section{Future Trends and Developments}
    \section{Conclusion}


    % % % % % % % % %
    % Bibliography  %
    % % % % % % % % %

    % \addcontentsline{toc}{section}{Works Cited}
    % \printbibliography
    \newpage
    \phantomsection % Create a phantom section to get the correct page number in the ToC
    \addcontentsline{toc}{section}{Works Cited}
    \printbibliography[title=Works Cited]

    % \bibliographystyle{plainnat}
    % \bibliography{sources}

\end{document}