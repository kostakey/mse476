\documentclass[12pt]{article}
\usepackage[letterpaper, top=1.25in, bottom=1.25in, left=1in, right=1in]{geometry}
\usepackage{fancyhdr} % http://ctan.org/pkg/fancyhdr
\usepackage{hyperref}
\usepackage{graphicx}
\usepackage{indentfirst}
\usepackage[export]{adjustbox}
\usepackage{textcomp}

% % % % % % % % % % % % 
% Citation formatting %
% % % % % % % % % % % % 

\usepackage[style=ieee]{biblatex}
\addbibresource{sources.bib}

% % % % % % % % %
% Hyperlinking  %
% % % % % % % % %

\hypersetup{
    colorlinks=false, % set true if you want colored links
    linktoc=all,     % set to all if you want both sections and subsections linked
    % linkcolor=black,  %c hoose some color if you want links to stand out
    % citecolor=black
    % urlcolor=black
}

% % % % % % % % % % % % % % % %
% Global title, author, date  %
% % % % % % % % % % % % % % % %

\title{Ferrous Metals in Electric Motors}
\author{Kosta Sergakis}
\date{December 12, 2025}
\makeatletter
\let\runauthor\@author
\let\runtitle\@title
\let\rundate\@date
\makeatother

% % % % % % % % % % %
% Header and Footer %
% % % % % % % % % % %

\pagestyle{fancy} % change page style to fancy
\fancyhf{} % clear header/footer
\setlength{\headheight}{15pt}
\fancyhead[L]{Sergakis}
\fancyhead[C]{\runtitle}
\fancyhead[R]{\rundate}
\fancyfoot[C]{\thepage} % \fancyfoot[R]{\thepage}
\renewcommand{\headrulewidth}{0.4pt} % default \headrulewidth is 0.4pt
\renewcommand{\footrulewidth}{0.4pt} % default \footrulewidth is 0pt

% % % % % % % % % %
% Paper Beginning %
% % % % % % % % % %

% \usepackage{natbib}
\begin{document}

    % % % % % % % %
    % Title Page  %
    % % % % % % % %
    
    \begin{titlepage}
        \begin{center}
            \vspace*{1.5cm}
            
            \textbf{Exploring Physical Metallurgy of Ferrous and Aluminum Alloys}
            
            \vspace{1cm}
            
            An exploration of:\\
            \vspace{0.25cm}
            \runtitle
                
            \vspace{1cm}
            
            \textbf{\runauthor}
            
            \vfill  
            
            \vspace{1cm}

            \includegraphics[width=0.3\textwidth]{resources/michigan-state-logo-png-transparent.png}
                
            Chemical Engineering and Materials Science\\
            Michigan State University\\
            East Lansing, Michigan\\
            \rundate
                
        \end{center}
    
    \end{titlepage}


    % % % % % % % % % % %
    % Table of Contents %
    % % % % % % % % % % %

    \setcounter{secnumdepth}{0} % removes section numbers
    % \maketitle
    
    \tableofcontents
    
    \newpage

    \section{Introduction}
        \subsection{Fundamentals of Electric Motors}
            
            \begin{center}
                \vspace{0.5cm}
                \includegraphics[width=0.35\textwidth]{resources/lorentz force.png}
                \\
                \textbf{Figure 1: Lorentz Force} \autocite{Hughes_2011}
                \label{lorentz}
            \end{center}
        
            Electric motors convert electrical energy into mechanical rotation through the interaction of magnetic fields and electric currents. When a current-carrying conductor is placed in a magnetic field, it experiences a Lorentz force, which motors harness to produce torque on a rotating rotor. \hyperref[lorentz]{[Figure 1]}. The stator provides a structured magnetic field, either through permanent magnets or electromagnetically energized windings, that drives rotor motion.
            
            \begin{center}
                \includegraphics[width=0.3\textwidth]{resources/dc commutator.png}
                \\
                \textbf{Figure 2: DC Motor Commutation} \autocite{5562688}
                \label{commutation}
            \end{center}

            \begin{center}
                \includegraphics[width=0.4\textwidth]{resources/ac motor.png}
                \\
                \textbf{Figure 3: AC Motor Rotating Magnetic Field} \autocite{Hughes_2011}
                \label{ac}
            \end{center}

            \begin{center}
                \includegraphics[width=0.85\textwidth]{resources/ac and dc motor types.png}
                \\
                \textbf{Figure 4: Motor Type Hierarchy} \autocite{ma17040848}
                \label{motor types}
            \end{center}

            % \begin{center}
            %     \includegraphics[width=0.45\textwidth]{resources/mechanical commutator.png}
            %     \\
            %     \textbf{Figure 5: Mechanical Commutation} \autocite{8086122}
            %     \label{mechanical commutation}
            % \end{center}

            In DC motors, commutators mechanically or electronically switch current direction in the rotor windings to maintain torque \hyperref[commutation]{[Figure 2]}. AC motors rely on alternating currents that naturally generate a rotating magnetic field in the stator, eliminating the need for brushes \hyperref[ac]{[Figure 3]}. As a result, DC machines offer simple speed control and low-cost operation, while AC machines provide higher durability, reduced maintenance, and better long-term reliability. Brushed DC architectures suffer from brush wear and friction, limiting speed capability, whereas AC induction and synchronous machines avoid these issues entirely. \hyperref[motor types]{Figure 4} shows the relationship between various motor types. 
            
            Speed regulation differs significantly between the two classes: DC motor speed is controlled by adjusting voltage or armature current, while AC motor speed is set by supply frequency using variable-frequency drives (VFDs). Although AC control is more complex, it provides greater efficiency and robustness in industrial systems.

            % \begin{center}
            %     \includegraphics[width=0.7\textwidth]{resources/typical dc and ac motor diagram.png}
            %     \\
            %     \textbf{Figure 5: Typical DC (a) and AC (b) Motor Diagram} \autocite{ma17040848}
            %     \label{motor diagram}
            % \end{center}

            \begin{center}
                \includegraphics[width=0.85\textwidth]{resources/spm vs ipm.png}
                \\
                \textbf{Figure 5: Surface Permanent Magnet (SPM) vs. Interior Permanent Magnet (IPM)} \autocite{ma17040848}
                \label{spm ipm}
            \end{center}

            \begin{center}
                \includegraphics[width=0.7\textwidth]{resources/axial vs radial.png}
                \\
                \textbf{Figure 6: Axial Flux vs. Radial Flux Motor Topology} \autocite{Gieras_Wang_Kamper_2008}
                \label{axial vs radial}
            \end{center}

            Among modern AC motor types, Interior Permanent Magnet (IPM) machines embed magnets within the rotor structure \hyperref[spm ipm]{[Figure 5]}. This geometry enhances air-gap flux density, increases torque production, improves high-speed mechanical strength, and can reduce energy consumption by roughly 30\% compared to surface-mounted permanent magnet (SPM) machines \autocite{ma17040848}. These benefits come with the tradeoff of higher manufacturing complexity and cost.

            Axial-flux permanent-magnet (AFPM) motors offer further advantages over traditional radial-flux machines, including higher power density, compact form factor, and improved cooling due to larger inner diameters \hyperref[axial vs radial]{[Figure 6]}. Advances in magnetic materials and thermal design have allowed AFPM topologies to exceed the power density limitations of conventional radial-flux architectures.

        \subsection{Applications Across Different Industries}

            \begin{center}
                \includegraphics[width=0.65\textwidth]{resources/power rating for motor types.png}
                \\
                \textbf{Figure 7: Power Rating for Motor Types} \autocite{Hughes_2011}
                \label{powah}
            \end{center}

            Different motor types such as brushed DC, BLDC, induction motors, and PMSMs, are selected based on required torque density, efficiency, speed range, cost, and environmental durability. Typical power ranges for these machines are shown in \hyperref[powah]{Figure 7}.  

            In automotive electrification, PMSMs with IPM rotors are more common due to their high torque density, excellent field-weakening capability for high-speed operation, and greater efficiency. Their embedded magnet design provides mechanical robustness at high RPM, improving EV range and performance. 

            Industrial automation relies heavily on BLDC and synchronous machines for their torque control, fast dynamic response, and low maintenance, making them ideal for robotics, CNC equipment, and conveyor systems. Induction motors remain the staple of HVAC and building systems due to their low cost, rugged construction, and compatibility with VFD-based energy-efficient speed control.

            In renewable energy, permanent-magnet generators in wind turbines benefit from high torque at low rotational speeds. Solar tracking systems frequently use sealed BLDC or stepper motors for reliable outdoor operation. Consumer devices, from drones and cooling fans to home appliances, increasingly use compact BLDC motors for their quiet operation and efficiency.

            Aerospace, medical, and defense sectors employ specialized BLDC and PMSM machines designed for high reliability, minimal electromagnetic interference, and operation in extreme environments. Weight, thermal management, and safety constraints drive highly optimized motor designs in these applications.

    \section{Materials Used in Electric Motors}
        \subsection{Introduction to Common Materials}
            
            \begin{center}
                \includegraphics[width=0.65\textwidth]{resources/material comparison.png}
                \\
                \textbf{Figure 8: Magnetic Flux Density - Magnetic Field Strength Comparison for Common Materials} \autocite{ma17040848}
                \label{material comparison}
            \end{center}

            Electric motors, particularly permanent-magnet motors, require a carefully selected combination of materials to ensure high performance, reliability, and efficiency. At the core, magnetic materials are critical; high-performance permanent magnets such as neodynium-iron-boron (Nd-Fe-B) and samarium-cobalt (Sm-Co) are used for their high energy product (BH), strong resistance to demagnetization, and thermal stability, while ferrites serve as a low-cost alternative for less demanding applications \hyperref[material comparison]{[Figure 8]} \autocite{ma17040848}. The motor's electromagnetic steel (typically laminated) must have low core losses, high magnetic permeability, and reduced eddy currents to maintain efficiency. Conductive materials like copper or aluminum are used for windings to provide high electrical conductivity, thermal stability, and mechanical strength under vibration and thermal cycling. 

            Structural materials are equally important: motor housings and frames require strength, corrosion resistance, and compatibility with thermal expansion, often achieved through aluminum or steel alloys, while shafts and bearings need high-strength, fatigue-resistant steels for long operational life and smooth rotation. 
            
            Electrical insulation in the windings and stator laminations must have high dielectric strength, thermal resistance, and chemical stability to prevent breakdown under high temperatures and operational stresses. Effective thermal management is also essential; materials with high thermal conductivity are used in heat sinks and cooling elements to dissipate heat efficiently, particularly in automotive and aerospace applications where weight is critical. Finally, the dependence on rare earth elements such as neodymium, dysprosium, and samarium in high-performance magnets introduces considerations of supply security, environmental impact, and cost, which are central to the sustainability and economic feasibility of motor production. Together, these material requirements ensure that electric motors meet the demanding mechanical, thermal, and electrical performance standards required in applications ranging from electric vehicles to industrial drives and robotics.

        \subsection{Permanent Magnet Characteristics}

            The BH concept refers to the relationship between magnetic flux density (B) and magnetic field strength (H) in a magnet, typically shown as a B-H curve. This curve illustrates how a magnetic material responds to an applied magnetic field, including how easily it becomes magnetized, how much magnetic energy it can store, and how well it resists demagnetization. A key metric derived from this curve is the maximum energy product, $(BH)_{max}$, which represents the peak value of the product, $B \times H$, on the demagnetization curve. This value effectively measures how much magnetic energy a permanent magnet can deliver per unit volume and is therefore a direct indicator of the magnet's strength. Materials like Nd-Fe-B and Sm-Fe-N are prized because their high $(BH)_{max}$ enables compact, lightweight, and efficient motor designs, making the BH concept central to understanding magnet selection and motor performance \hyperref[material comparison]{[Figure 8]}.

            \begin{center}
                \includegraphics[width=0.8\textwidth]{resources/hysteresis loop.png}
                \\
                \textbf{Figure 9: Hysteresis Loop} \autocite{Sterling}
                \label{hysteresis}
            \end{center}

            Remanent flux density, Br, is the magnetic flux density that remains in a magnet after an applied external magnetic field is removed. It represents how strongly the magnet stays magnetized without any external influence. A high Br means the material can maintain a strong magnetic field, which is desirable for motors because it allows high torque and power density. In a hysteresis loop (when the magnetic flux density in the rotor lags behind the magnetizing force, causing the rotor's magnetic poles to be constantly shifting), Br is the point where the magnetization curve crosses the vertical axis (B-axis) after saturation \hyperref[hysteresis]{[Figure 9]}. Essentially, Br represents how strong the magnet is when it is "left alone."

            Coercivity, Hc, is the amount of reverse magnetic field needed to reduce the magnet's flux density to zero after it has been saturated. This describes how resistant the magnet is to being demagnetized. Materials with high Hc can maintain their magnetization even under strong opposing magnetic fields, temperature changes, or load fluctuations. In motors, high coercivity is desirable, especially at elevated temperatures, because the magnetic field from the stator can weaken or destabilize the rotor magnet if coercivity is too low.

            Intrinsic coercivity, iHc, is a deeper measure of demagnetization resistance. It is the reverse field required to fully demagnetize the magnet's internal magnetization. In other words, iHc is the applied field strength needed to force the magnet's internal magnetic domains to zero. While Hc indicates when the external flux density crosses zero, iHc describes when the magnetization of the material itself becomes zero. 

            The Curie temperature, Tc, of a material is the temperature above which a ferromagnetic material loses its permanent magnetic properties. Below Tc, the magnetic domains within the material are aligned enough to produce strong, stable magnetization. This is what allows materials to serve as permanent magnets in motors. As the temperature rises, thermal energy disrupts the alignment of these domains. When the material reaches the Curie temperature, this disorder becomes so great that the material transitions from ferromagnetic (strong, permanent magnetism) to paramagnetic (weak, non-permanent magnetism). A magnet heated above its Tc cannot retain magnetic flux once cooled unless its remagnitized, and for many materials the loss is irreversible.
        
        \subsection{Permanent Magnet Materials}

            \begin{center}
                \includegraphics[width=0.78\textwidth]{resources/anisotropy vs polarization.png}
                \\
                \textbf{Figure 10: Anisotropy vs. Polarization} \autocite{COEY2020119}
                \label{anisotropy}
            \end{center}
            
            \subsubsection{Nb-Fe-B}

                \begin{center}
                    \includegraphics[width=0.55\textwidth]{resources/ndfeb magnetization curve.png}
                    \\
                    \textbf{Figure 11: NdFeB Magnetization Curve} \autocite{Ghezelbash2017}
                    \label{mag curve}
                \end{center}

                Nd-Fe-B is considered one of the best permanent-magnet materials because its intrinsic crystal chemistry, magnetic properties, and manufacturability align exceptionally well for producing very high-performance magnets. The main reason is due to the $Nd_2Fe_{14}B$ phase, a stable tetragonal compound that uniquely combines high saturation magnetization, strong magnetocrystalline anisotropy, and a reasonably high Curie temperature \hyperref[anisotropy]{[Figure 10]}. This phase provides a saturation magnetization of 95 emu/g, which is higher than most other rare-earth compound magnets \hyperref[mag curve]{[Figure 11]}. The saturation magnetization is higher than most due to its structure containing alternating Nd-rich layers and Fe-atom sheets that efficiently align and stabilize magnetic moments. Nd contributes large spin-orbit coupling, creating strong anisotropy, while Fe provides high magnetization \autocite{1063214}. 

                \begin{center}
                    \includegraphics[width=0.8\textwidth]{resources/ndfeb crystal structure.png}
                    \\
                    \textbf{Figure 12: $Nd_2Fe_{14}B$ Crystal Structure (a) and NdFeB Magnetic Structure Distribution (b)} \autocite{LIANG2024175689}
                    \label{ndfeb}
                \end{center}

                \begin{center}
                    \includegraphics[width=0.8\textwidth]{resources/sps vs. conventional.png}
                    \\
                    \textbf{Figure 13: NdFeB Fracture Surface After Spark Plasma Sintering (a) and After Conventional Powder Metallurgy (b)} \autocite{WANG20141}
                    \label{sps conv}
                \end{center}

                Microstructurally, Nd-Fe-B magnets benefit from a multi-phase structure created through sintering or powder metallurgy: a matrix of $Nd_2Fe_{14}B$ grains surrounded by thin Nd-rich grain-boundary phases \hyperref[ndfeb]{[Figure 12]}. These boundary phases help isolate grains magnetically and promote domain-wall pinning, increasing coercive force. Post-sintering heat treatments further enhance coercivity by reducing strain and improving grain-boundary uniformity \hyperref[sps conv]{[Figure 13]}. Nd-Fe-B is also highly "tunable." Substituting small amounts of Dy or Tb into the structure can boost coercivity (thanks to their higher anisotropy), while partial Fe-Co substitution increases the Curie temperature and improves thermal stability of magnetization \autocite{1063214}. This makes the material useful for high-temperature applications, though with some coercivity trade-off. 

                In short, Nd-Fe-B magnets are effective because their crystal structure naturally supports strong, directionally stable magnetism, and because metallurgical processing allows engineers to optimize coercivity, thermal stability, and microstructure for demanding modern applications.
            
            \subsubsection{Sm-Co}
                
                \begin{center}
                    \includegraphics[width=0.6\textwidth]{resources/smco mag.png}
                    \\
                    \textbf{Figure 14: $Sm_2Co_{17}$ (S1) and $Sm_2Co_{17}$ with More Fe (S2) Magnetization Curve} \autocite{7114284}
                    \label{smco mag curve}
                \end{center}

                Sm-Co magnets, particularly the $Sm_2Co_{17}$-type, are highly regarded among rare-earth magnets due to their magnetic properties, especially their high-temperature performance and corrosion resistance. The superior coercivity of these magnets originates primarily from domain wall pinning at the $SmCo_5$ cell boundary phase, which can be tuned by controlling Cu and Fe content during processing. The microstructure typically consists of a rhombohedral $Sm_2Co_{17}$ cell phase, hexagonal $SmCo_5$ boundary phase, and Zr-rich plate-like phases \autocite{7114284}. The Cu distribution at the cell boundaries and interfaces reduces the local magnetic anisotropy, creating energy wells that serve as strong attractive pinning sites for domain walls, which enhances coercivity. Additionally, variations in Fe content influences the saturation magnetization, with higher Fe content increasing the volume fraction of the cell phase and boosting magnetization \hyperref[smco mag curve]{[Figure 14]}. 

                The microstructure is highly stable even at elevated temperatures, enabling reliable operation well above 250-350\textdegree{}C, which is far beyond Nd-Fe-B's practical range. Sm-Co also possesses much higher Curie temperatures (720-860\textdegree{}C compared to ~310-400\textdegree{}C for Nd-Fe-B), giving it superior thermal stability compared to commercial permanent magnets \autocite{SINGH2015300}.

                \begin{center}
                    \includegraphics[width=0.85\textwidth]{resources/basal plane fracture.png}
                    \\
                    \textbf{Figure 15: Sintered $SmCo_5$ Fractographs in Two Loading Directions} \autocite{SINGH2015300}
                    \label{smco fract}
                \end{center}

                The differing appearances in the two loading directions highlight the material's strong crystallographic anisotropy: Loading parallel to the basal plane typically produces smoother, mirror-like cleavage, while loading perpendicular results in rougher, faceted features associated with crack deflection at grain boundaries and secondary phases \hyperref[smco fract]{[Figure 15]}.

                Mechanically, both Nd-Fe-B and Sm-Co are brittle, but Sm-Co is generally more fracture-prone due to its strongly anisotropic crystal structure \hyperref[anisotropy]{[Figure 10]}. The above fractograph reflects Sm-Co's crystallographic anisotropy and tendency toward cleavage-dominated failure \hyperref[smco fract]{[Figure 15]}. Nd-Fe-B's fracture surfaces tend to be less anisotropic and are more influenced by grain-boundary phases introduced through sintering. Sm-Co's brittleness complicates machining and limits its use in extremely high-RPM motors unless encapsulated, though its thermal robustness often outweighs these drawbacks.
                
                \begin{center}
                    \includegraphics[width=0.75\textwidth]{resources/cost.png}
                    \\
                    \textbf{Figure 16: Correlation of Crustal Abundance of the Elements and their Cost per Mole (a) and a Cost Periodic Table (b)} \autocite{6008648}
                    \label{cost}
                \end{center}

                From an application perspective, Nd-Fe-B dominates consumer electronics, EV traction motors, robotics, and most room-temperature industrial systems due to its high energy product and cost efficiency \hyperref[cost]{[Figure 16]}. Sm-Co, while more expensive, is preferred for aerospace, military, high-vacuum, high-radiation, and high-temperature machinery. In summary, Nd-Fe-B offers maximum magnetic strength but limited thermal resilience, whereas Sm-Co provides unmatched temperature stability, corrosion resistance, and magnetic reliability in extreme environments, making the two materials viable options in the permanent-magnet landscape.

    \section{Manufacturing of Electric Motors}

        \subsection{Effects of Hot Deformation}
            
            \begin{center}
                \includegraphics[width=0.7\textwidth]{resources/mfg process.png}
                \\
                \textbf{Figure 17: Production Process of Radially Oriented Ring Magnet and Axially Oriented Plate Magnet} \autocite{ma16196581}
                \label{mfg}
            \end{center}

            Hot-deformed Nd-Fe-B magnets are produced by melt-spinning nanocrystalline ribbons, pulverizing them into powder, hot-pressing to full density, and then hot-deforming to indice strong crystallographic texture and platelet grain orientation. This yields a fine-grained, high-performance magnet with a microstructure and domain behavior that is controlled by temperature, composition, and grain-boundary engineering \hyperref[mfg]{[Figure 17]}. 

            Hot deformation dramatically reshapes the grain structure and microstructure of Nd-Fe-B magnets, transforming them from an initially random, nanocrystalline state into a highly textured and anisotropic final structure. The process begins with melt spinning, which produces extremely fine 10-30nm $Nd_2Fe_{14}B$ grains that are randomly oriented due to rapid quenching. Pulverizing these ribbons into powder preserves this nanocrystalline structure while ensuring a slight excess of Nd-rich phases that will later form a liquid grain boundary. During hot pressing, the powder densifies fully and the grains coarsen to about 20-50nm, but the compact remains isotropic. The Nd-rich liquid phase distributes into a continuous grain boundary network that is essential for later grain movement. The most significant microstructural changes occur during hot deformation, where mechanical strain combined with the softened boundary phase allows grains to rotate, slide, and elongate. As a result, the originally equiaxed nanograins transform into plate-like grains. Typically, the grain aspect ratios lie between 200-500nm laterally and 20-50nm thick \autocite{ma16196581}. These platelets develop a strong crystallographic texture as their c-axes align along the compression direction. The grain boundary phase also reorganizes, forming thin, well-connected layers that promote domain-wall pinning. When properly controlled, the deformation step holds the nanoscale grain size, preventing abnormal grain growth while enhancing alignment. The final microstructure therefore consists of highly oriented, plate-like $Nd_2Fe_{14}B$ grains embedded in a continuous Nd-rich boundary network, producing a magnet with strong anisotropy, high remanence, and improved coercivity compared to conventional sintered magnets.
        
        \subsection{Effects of Sintering}

            Sintered Nd-Fe-B magnets are produced from rare-earth (RE) metals such as iron, cobalt, ferro-boron, and recycled alloy, starting with vacuum or argon induction melting to minimize oxygen contamination \autocite{Cui2022}. The microstructure of the cast alloy is critical, as the main magnetic phase, $Nd_2Fe_{14}B$, forms via a peritectic reaction with c-Fe. Slow cooling can leave soft, ductile a-Fe and secondary phases, reducing magnetic performance and complicating powder production \autocite{Cui2022}. Rapid cooling via strip casting suppresses a-Fe formation, produces thin sheets ($<$1mm), and enables higher remanence.

            \begin{center}
                \includegraphics[width=0.7\textwidth]{resources/conventional vs hd.png}
                \\
                \textbf{Figure 18: Conventional Sintered Process and Hot-Press and Hot-Deformation Process} \autocite{ma16134789}
                \label{sint vs hddr}
        \end{center}

        \begin{center}
                \includegraphics[width=0.7\textwidth]{resources/mfg effects.png}
                \\
                \textbf{Figure 19: Coercivity vs. Grain Size (a), Typical Microstructural Features of Sintered Nd-Fe-B Magnets (b), HDDR Nd-Fe-B Powders (c), and Melt-Spun Nd-Fe-B Ribbons (d)} \autocite{ma16134789}
                \label{coerc vs grains}
        \end{center}

            The strip-cast flakes undergo hydrogen decrepitation, creating microcracks that embrittle the material, followed by jet milling to produce a signle-crystal powder (about 5\textmu{}m) with uniform particle size \hyperref[coerc vs grains]{[Figure 19]}. The powder is pressed under a magnetic field, aligning grains for anisotropic properties \hyperref[sint vs hddr]{[Figure 18]}.

            Sintering occurs under vacuum or inert gas, densifying the magnet ($>$98\% theoretical density) while controlling grain growth. Post-isothermal heat treatment optimizes coercivity and B-H loop shape. Pressed magnets shrink anisotropically (7-25\%), which requires machining (wire EDM, grinding, or vibratory honing) to achieve precise dimensions and reduce chipping \autocite{Cui2022}.

            Nd-Fe-B magnets require thin corrosion-resistant coatings to prevent degradation, and are magnetized post-processing. To improve high-temperature performance, heavy rare-earth (Dy, Tb) diffusion or additions are used to maintain coercivity above 150\textdegree{}C.

            Sintered Nd-Fe-B magnets dominate the market due to their mature manufacturing processes and excellent performance, though they require significant amounts of expensive heavy rare-earth (HRE) elements, such as dysprosium or terbium, to maintain coercivity at elevated temperatures above 150\textdegree{}C. 

    \section{Conclusion}
        
        Electric motors rely on the selection of materials and the precision of manufacturing processes to achieve high performance, efficiency, and reliability. Permanent magnets, particularly Nd-Fe-B and Sm-Co, play a central role in enhancing motor power density and torque characteristics due to their strong magnetic properties. Nd-Fe-B magnets offer high remanence and energy density but require careful microstructural control through processes such as strip casting, hydrogen decrepitation, sintering, and heat treatment to optimize coercivity and alignment. Sm-Co magnets, while less sensitive to temperature, provide excellent thermal stability and corrosion resistance.

        Manufacturing techniques, including hot deformation for Nd-Fe-B and sintering for both magnet types, directly influence grain orientation, texture, and the formation of grain-boundary phases, which in turn determine magnetic performance. The extra machining, coating, and magnetization steps are critical for producing magnets suitable for practical applications across industries ranging from automotive to renewable energy. 

        Overall, the combination of advanced materials, microstructural engineering, and precise manufacturing processes enable electric motors to meet increasingly demanding performance requirements, supporting the evolution of electrification in modern technology.


    % % % % % % % % %
    % Bibliography  %
    % % % % % % % % %

    % \addcontentsline{toc}{section}{Works Cited}
    % \printbibliography
    \newpage
    \phantomsection % Create a phantom section to get the correct page number in the ToC
    \addcontentsline{toc}{section}{Works Cited}
    \printbibliography[title=Works Cited]

    % \bibliographystyle{plainnat}
    % \bibliography{sources}

\end{document}